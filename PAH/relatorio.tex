\documentclass[11pt,a4paper]{article}
\usepackage{tikz}
\usetikzlibrary{arrows,shapes}
\usepackage[utf8]{inputenc} 
\usepackage{setspace}
\usepackage{indentfirst}
\usepackage{graphicx}
\usepackage{color}
\usepackage{verbatim}
\pagenumbering{arabic}
\usepackage{a4wide}
\usepackage{biblatex}
\onehalfspacing
\usepackage[pdftex]{hyperref}
\begin{document}




\title{ Relatório PAH }
\author{ André Santos, Helena Alves, Pedro Carneiro }
\date{ --/--/2013 }




\maketitle
\newpage
\tableofcontents
\newpage


 
\section{Introdução}
Linguagens como o LaTex e o Html trazem grandes benefícios para ..., contudo também possuem alguns pontos menos atraentes. Estes pontos dizem respeito a certas desvantagens das mesmas, nomeadamente, a complexidade de certos comandos. Além de outros aspectos, este foi um dos aspectos que levou à criação da ferramenta \textit{PAH}. Uma maior facilidade na escrita de certos comandos leva a um maior desempenho por parte do programador, aumentado o seu tempo de trabalho. Deste modo, a simplificação de certos comandos foi um dos aspectos principais a serem melhorados. 
Esta nova linguagem não acarreta, de longe, todas as potencialidades e funcionalidades das duas linguagens em causa. No entanto, foram trabalhadas grande parte das funcionalidades mais usuais, tais como, a introdução de capítulos, texto formatado e não formatado, formatação de tabelas e listas, inclusão d eimagens e links, introdução de comentários, geração de índices, entre outros. bla bla bla......


 \newpage 
\section{Composicão e Instalação}
A ferramenta PAH está desenvolvida em flex, sendo esta constituída pelos seguintes ficheiros:

\begin{description}
    \item[latex.fl:] código fonte da ferramenta para gerar latex;
    \item[html.fl:] código fonte da ferramenta para gerar html;
    \item[pah.c:] main do programa;
    \item[functionsLatex.c:] funções em C usadas pelo ficheiro latex.fl;
    \item[functionsHtml.c:] funções em C usadas pelo ficheiro html.fl;
    \item[linkedlist.c:] módulo de listas ligadas;
    \item[utilities.c:] 
    \item[makefile:] cria o respectivo executável.
\end{description}
 

Para proceder à instalação da ferramenta, basta apenas copiar os ficheiros referidos anteriormente para uma directoria à escolha e executar a makefile. Desta forma, será criado um executável na mesma directoria, com o nome de \textit{pah}.


\section{Invocação}
Como um dos objectivos deste pré-processador é facilitar a conversão em latex e html, a invocação para gerar ficheiros para cada uma destas linguagens consiste em: 

\begin{description}
    \item[Gerar LaTeX:] ./pah --latex ficheiro ficheiro.tex
    \item[Gerar HTML:] ./pah --html ficheiro ficheiro.html
\end{description}


\textbf{Outras situações:}
\begin{itemize}
    \item \textbf{Gera ficheirotex.tex e ficheirohtml.html:} 
    ./pah ficheiro ficheirotex.tex ficheirohtml.html
    \item \textbf{Gera ficheiro.tex e ficheiro.html} 
    ./pah ficheiro ficheiro
    \item \textbf{Gera Latex.tex e Html.html:}
    ./pah ficheiro
    \item \textbf{Erro por falta de ficheiro de input:}
    ./pah --latex
    ./pah --html
    ./pah
\end{itemize}


 \newpage 

\section{Funcionalidades da ferramenta}
Nas secções seguintes serão apresentadas e exemplificadas todas as funcionalidades oferecidas por parte desta nova ferramenta. 
 

\subsection{Cabeçalhos do documento}
Num cabeçalho de um documento é possível encontrar título, autores e data de um documento. Como seria de esperar, a ferramente \textit{PAH} também proporciona esta funcionalidade, baseando-se nos seguintes comandos:

\begin{itemize}
    \item \textbf{Título do documento:} 
        \verb@ %title Título @
    \item \textbf{Autor do documento:}
        \verb@ %author Autor @
    \item \textbf{Data do documento}
        \verb@ %date 02/04/2013 @
\end{itemize}

Além disso, para que seja possível que o documento possua uma capa, basta usar o comando \verb@ %cover% @, que em LaTeX diz respeito ao comando \verb@ \maketitle @, e em Html diz respeito a uma zona do documento onde terá toda esta informação.


\subsection{Índices}
Para um índice de um documento ser gerado, basta introduzir o comando:

\begin{itemize}
    \item \verb@ %index% @
\end{itemize}

Este comando deve ser colocado após a capa do documento e antes de qualquer conteúdo.
 

\subsection{Comandos em LaTeX e HTML}
Em qualquer momento, a introdução de comandos nativos de LaTeX/HTML pode ser necessária, de forma a cobrir as funcionalidades que a ferramente %it{PAH(null) não proporciona. Desta forma, para ambas as linguagens, a introdução destes comandos é feita de forma muito simples, bastando para isso seguir os seguintes comandos:

 \begin{itemize}
    \item \textbf{Para LaTeX:} 
        \verb@ %%latex% Comandos nativos em latex %latex%% @
    \item \textbf{Para HTML:}
        \verb@ %%html% Comandos nativos em html %html %% @
\end{itemize}


\subsection{Títulos do documento}
Existem três tipos de títulos, sendo estes apresentados pela ordem de maior importância.

\begin{itemize}
    \item \verb@ %t1{Título da secção} @
    \item \verb@ %t2{Título da subsecção} @
    \item \verb@ %t3{Título de subsubsecção} @
\end{itemize}

\verb@ %t2{Título de subsubsecção} @

 
\subsection{Texto Não Formatado}
Uma das funcionalidades da ferramenta é permitir incluir texto que não seja interpretado pelo %it{PAH(null). Desta forma, todo o texto compreendido entre os comandos \verb@ %%verbatim% e %verbatim%% @, não será interpretado aquando da compilação, sendo este apresentado no documento sem quaisquer modificações. Este último caso é utilizado aquando da introdução de várias linhas. Caso o pretendido seja apenas a introdução de uma linha, basta introduzir o conteúdo dessa linha entre os comandos \verb@ %%verb% e @ %verb%%.


\subsection{Texto Formatado}
Ao longo do documento, partes do texto podem ser apresentadas em itálico, em negrito, ou até mesmo sublinhadas. Para isso, basta seguir os próximos comandos:

\begin{description}
    \item[Itálico:] %%verb% \textit{Texto em itálico} %verb%%
    \item[Negrito:] %%verb% \textbf{Texto em negrito} %verb%%
    \item[Sublinhado:] %%verb% \underline{Textto sublinhado} %verb%%
\end{description}


\subsection{Listas} 
De seguida, são apresentadas os três tipos de listas suportadas pela ferramenta %it{PAH(null).
 

\subsubsection{Listas Numeradas}

\begin{itemize}
    \item Iniciada por \verb@ %%ol% @ e finalizada por \verb@ %ol%% @;
    \item Cada item da lista é precedido por \verb@ %it @;
\end{itemize}


\subsubsection{Listas Não Numeradas}

\begin{itemize}
    \item Iniciada por \verb@ %%ul% @ e finalizada por \verb@ %ul%% @;
    \item Cada item da lista é precedido por \verb@ %it @;
\end{itemize}


\subsubsection{Entradas de um dicionário}

\begin{itemize}
    \item Iniciada por \verb@ %%dl% @ e finalizada por \verb@ %dl%% @;
    \item Cada item é precedido por \verb@ %it{} @ e o texto que ficará a negrito deverá ser colocado entre os parêntesis;
    \item A descrição do item deve ser colocada após o fim dos parêntesis.
\end{itemize}

\textbf{Exemplo:}

\begin{verbatim}
%%dl%
    %it{Item} descrição do item
    %it{Item} descrição do item
%dl%%
\end{verbatim} 


\subsection{Imagens}
Como qualquer outra ferramenta, é permitida a inclusão de imagens no documento. Para isso, basta usar o comando \verb@ %img{path,caption} @, sendo "path" o caminho para a respectiva imagem e "caption" a legenda da mesma.

\subsubsection{Exemplo}
\verb@ %img{exemplo.png, Legenda da imagem} @


\subsection{Links}
A presença de links num documento é permitida usando o comando \verb@ %link{url,texto} @. No campo "texto" é inserido o texto que fica associado ao link colocado no campo "url".
 

\subsection{Comentários}
O uso de comentários é muito usual ao longo do texto, uma vez que facilitam a leitura do mesmo e não aparecem no documento final. No caso de o comentário abrangir várias linhas, estas devem ser colocadas entre os comandos \verb@ %%coment% e %coment%% @. Por outro lado, se o pretendido for apenas um comentário de linha, coloca-se o texto a comentar após o comando \verb@ %%% @.

\begin{itemize}
    \item \verb@ %%% Isto é um comentário de linha @
\end{itemize}


\subsection{Tabelas} 
As tabelas têm a seguinte estrutura:



            
            

\begin{table}[!htpb]
\centering
\begin{tabular}{||c|c||}
\hline
 \textbf{ Nome   } & \textbf{  Número  } \\
\hline
  Helena Alves   &  a61000 \\
\hline
\end{tabular}
\end{table}



\begin{verbatim}
    %%table%
            %head Célula  % Célula 
            %row  Célula  % Célula  
    %table%%
\end{verbatim}


\begin{itemize}
    \item A tabela é iniciada com \verb@ %%table% @ e finalizada com \verb@ %table%% @;
    \item Cada célula da tabela deve ser colocada após \verb@ %head @ ou \verb@ %row @. A única diferença entre estes, consiste no formato do texto. Enquanto que o comando \verb@ %head @ coloca o texto centrado e a negrito, o comando \verb@ %row @ coloca o texto num formato normal.
    \item O comando \verb@ % @ permite a separação das células da tabela, correspondendo, no entanto, ao número total de colunas.
\end{itemize}


A introdução de tabelas é um exemplo dos objectivos da nova ferramenta, nomeadamente, a simplificação dos comandos. Neste caso, tratando-se da introdução de tabelas, é necessário apenas a introdução dos comandos acima referidos, tornando a sua introdução muito mais facilitada, diminuindo drasticamente a sua complexidade. 

 \newpage 

\subsection{Conclusão}

neste lfgfldkjjdk
\end{document}
